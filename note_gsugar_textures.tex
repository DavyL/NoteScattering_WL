% !TEX TS-program = pdflatex
% !TEX encoding = UTF-8 Unicode

% This is a simple template for a LaTeX document using the "article" class.
% See "book", "report", "letter" for other types of document.

\documentclass[11pt]{article} % use larger type; default would be 10pt

\usepackage[utf8]{inputenc} % set input encoding (not needed with XeLaTeX)

%%% Examples of Article customizations
% These packages are optional, depending whether you want the features they provide.
% See the LaTeX Companion or other references for full information.

%%% PAGE DIMENSIONS
\usepackage{geometry} % to change the page dimensions
\geometry{a4paper} % or letterpaper (US) or a5paper or....
% \geometry{margin=2in} % for example, change the margins to 2 inches all round
% \geometry{landscape} % set up the page for landscape
%   read geometry.pdf for detailed page layout information

\usepackage{graphicx} % support the \includegraphics command and options

% \usepackage[parfill]{parskip} % Activate to begin paragraphs with an empty line rather than an indent

%%% PACKAGES
\usepackage{booktabs} % for much better looking tables
\usepackage{array} % for better arrays (eg matrices) in maths
\usepackage{paralist} % very flexible & customisable lists (eg. enumerate/itemize, etc.)
\usepackage{verbatim} % adds environment for commenting out blocks of text & for better verbatim
\usepackage{subfig} % make it possible to include more than one captioned figure/table in a single float
% These packages are all incorporated in the memoir class to one degree or another...

\usepackage{hyperref}
\usepackage{amsmath, amssymb}
\usepackage{bbm}
%%% HEADERS & FOOTERS
\usepackage{fancyhdr} % This should be set AFTER setting up the page geometry
\pagestyle{fancy} % options: empty , plain , fancy
\renewcommand{\headrulewidth}{0pt} % customise the layout...
\lhead{}\chead{}\rhead{}
\lfoot{}\cfoot{\thepage}\rfoot{}

%%% SECTION TITLE APPEARANCE
\usepackage{sectsty}
\allsectionsfont{\sffamily\mdseries\upshape} % (See the fntguide.pdf for font help)
% (This matches ConTeXt defaults)

%%% ToC (table of contents) APPEARANCE
\usepackage[nottoc,notlof,notlot]{tocbibind} % Put the bibliography in the ToC
\usepackage[titles,subfigure]{tocloft} % Alter the style of the Table of Contents
\renewcommand{\cftsecfont}{\rmfamily\mdseries\upshape}
\renewcommand{\cftsecpagefont}{\rmfamily\mdseries\upshape} % No bold!

%%% END Article customizations

%%% The "real" document content comes below...

\title{Note log leaders et minimisation de covariance}
\author{Leo Davy}
%\date{} % Activate to display a given date or no date (if empty),
         % otherwise the current date is printed 

\begin{document}
\maketitle
\begin{equation}
	\hat{x}(y;\Lambda) \in Arg\min_{x} ||y - \Phi x||_{W}^2 + ||U_\Lambda x||_q^q
\end{equation}

\section{log-leaders}
\begin{equation}
	l_{j,n} = lg(\mathcal{L}_{j,n}) = v_n + jh_n + \zeta_{j,n}
\end{equation}
où $\zeta_{j,n}$ suit une loi normale centrée (de variance inconnue (?)).
Par hypothèse, si $n_1$ et $n_2$ sont de la même texture, alors $v_{n_1}=v_{n_2}$ et $h_{n_1} = h_{n_2}$.
\begin{equation}
	(\hat{h}, \hat{v}) = arg\min_{h,v} \sum_{j_1}^{j_2} ||jh + v - l_j||_2^2 \quad (+\lambda TV(v,h))
\end{equation}
Au sein d'une même texture, $h$ et $v$ vont tendre vers la moyenne de $l_j$, mais en tout point, $jh_n + v_n - l_{j,n}$ suit une loi normale, même si $h_n$ et $v_n$ ont exactement les bonnes valeurs.
\newline
N'est-ce pas plus précis, une fois que l'on sait qu'un ensemble de points fait partie de la même texture de remplacer $l_{j,n}$ par la valeur moyenne de $l_j$ sur cette texture ? 

\section{Minimisation de la "covariance" dans une texture}
En définissant la "covariance" intra texture
\begin{equation}
	Cov(\Omega_{m,j}) =  \frac{1}{|\Omega_m|^2}\sum_{n_1,n_2\in\Omega_m} (jh_m + v_m  - l_{n_1,j})(j h_m + v_m - l_{n_2, j})
\end{equation}
où $\Omega_m$ est un ensemble de points.
\newline
Est-ce que la segmentation de textures, donne une partition de l'ensemble de points $\Omega_1, \cdots, \Omega_M$ qui minimise cette quantité ?
\newline
La minimisation peut être dans le sens suivant : $\forall n_i \in \Omega_i, n_i \not\in\Omega_m$, alors
\begin{equation}
	Cov(\Omega_m\cup \{n_i\}) + Cov(\Omega_i \backslash \{n_i\}) \geq Cov(\Omega_m) + Cov(\Omega_i)
\end{equation}


\end{document}
